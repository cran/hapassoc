\documentclass[article, shortnames]{jss}
\usepackage{graphicx}
%% \VignetteIndexEntry{hapassoc}

%%%%%%%%%%%%%%%%%%%%%%%%%%%%%%
%% declarations for jss.cls %%
%%%%%%%%%%%%%%%%%%%%%%%%%%%%%%

%% just as usual
\author{Kelly Burkett\\Simon Fraser University \And
        Jinko Graham\\Simon Fraser University \And 
	Brad McNeney\\Simon Fraser University  
	}
\title{\pkg{hapassoc}:\\ Software for likelihood inference of trait associations with SNP haplotypes and other attributes}

%% for pretty printing and a nice hypersummary also set:
\Plainauthor{Kelly Burkett} %% comma-separated
\Plaintitle{Hapassoc: Software for likelihood inference of trait associations with SNP haplotypes and other attributes} %% without formatting
\Shorttitle{} %% a short title (if necessary)

%% an abstract and keywords
\Abstract{
Complex medical disorders, such as heart disease and diabetes, are thought
to involve a number of genes which act in conjunction with lifestyle and
environmental factors to increase
disease susceptibility.
Associations between complex traits and single nucleotide polymorphisms 
(SNPs) in candidate genomic regions can provide a useful tool for 
identifying genetic risk factors. However, analysis of trait 
associations with single SNPs ignores the potential for extra information 
from haplotypes, combinations of variants at multiple SNPs 
along a chromosome inherited from a parent.  
When haplotype-trait associations are of interest and haplotypes of 
individuals can be determined, generalized linear models (GLMs) may be used 
to investigate haplotype associations 
while adjusting for the effects of non-genetic cofactors or attributes. 
Unfortunately, haplotypes cannot always be determined cost-effectively 
when data is collected on unrelated subjects. 
Uncertain haplotypes may be inferred on the basis of data from single SNPs.
However, subsequent analyses of risk factors must account for the resulting 
uncertainty in haplotype assignment in order to avoid potential errors in 
interpretation. To account for such uncertainty, we have developed
\pkg{hapassoc}, software for \proglang{R} implementing a 
likelihood approach to inference of haplotype and non-genetic effects 
in GLMs of trait associations. We provide a description of the underlying 
statistical method and illustrate the use of \pkg{hapassoc} with examples 
that highlight the flexibility to specify dominant and recessive effects 
of genetic risk factors, a feature not shared by other software that 
restricts users to additive effects only. Additionally, \pkg{hapassoc} 
can accommodate missing SNP genotypes for limited numbers of subjects.
}
\Keywords{genetic association, single-nucleotide polymorphisms,
haplotypes, generalized linear models, missing data, EM algorithm}
%%at least one keyword, comma-separated, not capitalized

%% publication information
%% NOTE: This needs to filled out ONLY IF THE PAPER WAS ACCEPTED.
%% If it was not (yet) accepted, leave them commented.
\Volume{16}
\Issue{2}
\Month{May}
\Year{2006}
\Submitdate{2005-10-20}
\Acceptdate{2006-04-26}

%% The address of (at least) one author should be given
%% in the following format: 
\Address{
Kelly Burkett\\
Department of Statistics and Actuarial Science \\
Simon Fraser University \\
8888 University Drive \\
Burnaby BC, Canada,V5A 1S6 \\
E-mail: kburkett@sfu.ca\\
URL: http://stat-db.stat.sfu.ca:8080/statgen\\
}
%% It is also possible to add a telephone and fax number
%% before the e-mail in the following format:
%% Telephone: +43/1/31336-5053
%% Fax: +43/1/31336-734

%% for those who use Sweave please include the following line (with % symbols):
%% need no \usepackage{Sweave.sty}

%% end of declarations %%%%%%%%%%%%%%%%%%%%%%%%%%%%%%%%%%%%%%%%%%%%%%%


\begin{document}

\section{Introduction and Background}


The identification of genetic factors influencing susceptibility to
complex diseases such as cancer and diabetes is important for improving
our understanding of disease pathways. Genetic factors are measured at
multiple sites of known genomic location (\emph{genetic markers}) 
to determine if variants at these sites
are associated with the disease trait.
A common type of genetic marker in association studies is the
\emph{single nucleotide polymorphism} (SNP). SNPs
generally have only two different variants, called
\emph{alleles}, which can be labelled as 0 or 1. For each marker, an
individual inherits one allele from their mother, and one from their
father. The \emph{genotype} is the particular two alleles that are
inherited. The three possible genotypes for a SNP marker are 0/0, 0/1 and 1/1, 
where ``/'' is used to separate the alleles inherited from each parent. 
The 0/0 and 1/1 genotypes are called \emph{homozygous} (both alleles are the same)
and the 0/1 genotype is called \emph{heterozygous} (the two alleles are different).

A \emph{haplotype} denotes the alleles at multiple
markers that are inherited together on the same chromosome from a parent.  
For example, if an individual inherits a 0 allele at 
one SNP and a 0 allele at another SNP  
from their mother, the haplotype they inherit is 00.  
Because the SNPs may be close together on the same 
chromosome, their alleles are not necessarily independent.
The \emph{haplotype phase} is the combination of the two 
haplotypes that an individual inherits. For example, 
the phased genotype 00/01 indicates 
that the two haplotypes inherited were 00 and 01; the genotype at 
the first SNP is 0/0 and the genotype at the second SNP is 0/1.  

When associations between haplotypes and disease outcomes or \emph{traits} are
of interest, a potential difficulty is that haplotype phase is not 
necessarily known because, typically, genotypes are only measured at 
individual markers. For a subject who is heterozygous at one or fewer
markers, the haplotype phase can be inferred from the 
genotypes at individual markers. However, for a subject who is heterozygous at 
two or more markers, the phase can not be determined with certainty. 
This is illustrated in figure \ref{fig.unknown.phase}. 
Current cost-effective genotyping technology gives the observed genotypes 
at the two SNPs as (0/1, 0/1). The observed genotypes could involve the 
inheritance of haplotypes 00 and 11, or of haplotypes 01 and 10.
Therefore, two possible haplotype phasings,
00/11 and 01/10, are compatible with the observed data.

\begin{figure}[htp]\label{fig.unknown.phase}
\begin{center}
\begin{picture}(260,130)

%First pedigree
\put(20,111){\circle{30}}
%square
\put(70,125){\line(1,0){30}}
\put(70,95){\line(1,0){30}}
\put(70,95){\line(0,1){30}}
\put(100,95){\line(0,1){30}}
\put(53,55){\circle{30}}

\put(85,95){\line(-1,-1){25}}
\put(20,95){\line(1,-1){25}}
\put(20,75){00}
\put(75,75){11}
\put(40,27){00/11}
\put(0,10){\bf Observed Genotypes}
\put(35,0){\bf 0/1; 0/1}

%Second pedigree
\put(175,111){\circle{30}}
%square
\put(225,125){\line(1,0){30}}
\put(225,95){\line(1,0){30}}
\put(225,95){\line(0,1){30}}
\put(255,95){\line(0,1){30}}
\put(208,55){\circle{30}}

\put(240,95){\line(-1,-1){25}}
\put(175,95){\line(1,-1){25}}
\put(175,75){01}
\put(230,75){10}
\put(195,27){01/10}
\put(155,10){\bf Observed Genotypes}
\put(190,0){\bf 0/1; 0/1}

\end{picture}
\caption{Illustration of unknown haplotype phase for certain observed genotypes}
\end{center}
\end{figure}

The analysis of haplotype-trait associations therefore involves 
handling the missing phase data. To describe such associations, 
\citet{Lake03} and \citet{Burkett04} adopt a \emph{generalized linear model} 
(GLM) framework \citep{McCullaghNelder83}. GLMs provide the flexibility to 
incorporate non-genetic risk factors or potential confounding variables such 
as age or sex as covariates. They also allow the incorporation of interaction 
between genetic and environmental risk factors, a current research focus in 
the study of complex diseases. Finally, GLMs can accommodate a variety of disease
 outcomes including dichotomous, count and continuous traits. The method of 
\citet{Burkett04} is implemented in \pkg{hapassoc}, a contributed package for 
\proglang{R}. Standard population- and quantitative-genetic assumptions,
such as population Hardy-Weinberg equilibrium and independence of 
haplotypes and non-genetic covariates, are made. Parameter estimates are 
obtained by use of an expectation-maximization (EM) algorithm, called the 
method of weights \citep{Ibrahim90}, for missing categorical covariates. 
Standard errors that account for the extra uncertainty due to missing phase 
are calculated using Louis' method \citep{Louis82}. The statistical approach 
implemented in \pkg{hapassoc} is briefly described and the use of the program is
illustrated with examples.

\section{Statistical Description}

Let $y_i$ be the measured trait and $x_i$ be the vector of covariates for
the $i$th subject, assuming known haplotype phase. The covariates $x_i$
provide information about the individual's haplotypes and non-genetic
cofactors. Let $\Prob(x_i \mid \gamma)$ be the probability of covariates
$x_i$, parametrized by $\gamma$.  For notational convenience, suppose $\gamma$ is
a column-vector. A GLM specifies the conditional probability of the
trait given the covariates as $$\Prob(y_i \mid x_i, \beta, \phi) = \exp \left
[ \frac{y_i \nu_i - b(\nu_i)}{a_i(\phi)} + c(y_i, \phi) \right ],$$ where
$\nu_i$ is the canonical parameter for a probability distribution in the
exponential family, $\phi$ is a dispersion parameter, $a_i(\phi)=\phi/m_i$
for a known constant $m_i$, and $b(\cdot)$ and $c(\cdot,\cdot)$ are known
functions. A link function $g(\mu_i)=\eta_i$ relates the linear model
$\eta_i = x_i^T \beta$ to the mean trait value $\mu_i=b'(\nu_i)$. The
regression coefficients $\beta$ describe the effects of haplotypes and
non-genetic cofactors, including possible interactions, on the mean trait
value. The parameters that describe the trait and covariate data are thus
$\theta = (\beta^T, \phi, \gamma^T)^T$. However, we only consider $\phi$ and
$\gamma$ as needed for inference about $\beta$, which is of primary
interest.

If haplotype phase were known and complete covariate information were
available on all subjects, the complete-data log-likelihood for the $i$th
individual would be $$l_{ci}(\theta)=\log \left [ \Prob (y_i \mid x_i, \beta,
\phi) \times \Prob(x_i \mid \gamma) \right ].$$ Haplotypes are not necessarily
known, and therefore for some subjects the genetic components of the
covariate vector $x_i$ are not available. However, the observed 
genotypes at individual SNPs constrain the potential haplotype configurations. 
Hereafter, let $x_{obs,i}$ be the observed covariate information for
the $i$th subject and $x_i^j$ be the $j$th possible covariate vector 
(i.e. covariates that code haplotype information 
and non-genetic covariates) consistent with $x_{obs,i}$.

We account for missing haplotype phase by use of the EM algorithm
\citep{Dempster77}, in which maximum likelihood estimates $\widehat
\theta$ are obtained by iteratively maximizing the conditional expectation
of the complete-data log-likelihood, $Q(\theta \mid \theta_t)$, given the
observed data and current parameter estimates $\theta_t$ to determine new
parameter estimates $\theta_{t+1}$. Assuming independent subjects,
$Q(\theta \mid \theta_t) = \sum_i Q_i (\theta \mid \theta_t)$, 
where $Q_i (\theta \mid \theta_t) = 
{\E} \left [ l_{ci}(\theta) \mid 
y_i, x_{obs,i}, \theta_t \right ]$.

Our EM algorithm is based on the method of weights \citep{Ibrahim90}, an
implementation for generalized linear models with categorical covariates
that may be missing \citep[see][for a review]{HortonLaird99}. 
Let $l_{ci}^j(\theta)=\log \Prob(y_i, x_i^j \mid \theta)$
be the complete-data log-likelihood for the $i$th subject
if that subject had covariate vector $x_i^j$. Then 
$$Q_i (\theta \mid \theta_t)= \sum_j l_{ci}^j(\theta)
\Prob(x_i^j \mid y_i, x_{obs,i}, \theta_t) 
= \sum_j l_{ci}^j(\theta ) \; w_{ij}(\theta_t), $$
where $w_{ij}(\theta_t)=\Prob(x_i^j \mid y_i, x_{obs,i}, \theta_t)$ are the
``weights''. 
Note that each subject's contribution $Q_i(\theta \mid \theta_t)$ to $Q(\theta
\mid \theta_t)$ is summed over the set of covariate vectors consistent
with their observed data ($y_i,x_{obs,i}$). The EM algorithm involves 
computing weights for each possible complete-data covariate vector (E-step), 
and computing new parameter estimates by maximizing a weighted 
log-likelihood function (M-step). Upon convergence, standard errors can be 
calculated.  We implemented this EM algorithm and the calculation of 
standard errors in \proglang{R} (http://www.r-project.org) as described 
in detail below. Our implementation is freely available on the Comprehensive 
R Archive Network (CRAN) as a package called \pkg{hapassoc}.
  
\subsection[EM algorithm implemented in hapassoc]{EM algorithm implemented in \pkg{hapassoc}}

Our implementation of the algorithm is summarized by the following steps.

\begin{enumerate}

\item For each individual with unknown haplotypes, add a ``pseudo-individual'' 
for each possible haplotype combination consistent with that individual's 
observed genotypes. The genetic data for each pseudo-individual is a different 
haplotype configuration; the non-genetic data is the same for each 
pseudo-individual. Initial values for regression coefficients and haplotype 
frequencies are set.

\item At the $t$th iteration:

\begin{enumerate}

\item E-step: Compute the expected log-likelihood conditional on the
observed data and the current parameter estimates
$\theta_t = (\beta_t^T,\phi_t, \gamma_t^T)^T$.

This step simplifies to weighting each pseudo-individual by 
the conditional probability of the corresponding haplotype configuration given 
the observed data:
$$
w_{ij}(\theta_t)  =  
\Prob(x_i^j \mid y_i,x_{obs,i} \; \theta_t) 
 = 
\frac{\Prob(y_i\mid x_i^j, \beta_t, \phi_t) \Prob(x_i^j \mid \gamma_t)}{
\sum_{k}\Prob (y_i\mid x_i^k, \beta_t, \phi_t) \Prob(x_i^k \mid \gamma_t)}
. $$

Calculation of $\Prob(y_i \mid x_i^j , \beta_t, \phi_t)$ is 
straightforward using the GLM and the estimates $\beta_t$ and $\phi_t$
from fitting the model in the M-step at iteration $t$.
To specify the covariate distribution $\Prob (x_i^{j} |\ \gamma_t)$,
two standard assumptions from the genetics literature
are made. The first is independence of the 
genetic and non-genetic factors:
$$\Prob (x_i^{j} |\ \gamma_t)=\Prob(x_{gi}^{j} |\ \gamma_t) 
\Prob(x_{ei}^{j}|\ \gamma_t),$$ where $x_{gi}^{j}$ 
and $x^j_{ei}$ code, respectively, the haplotype and non-genetic covariate 
information. If only the haplotype information is missing, this assumption 
means that 
$$w_{ij}(\theta_t) 
 = 
\frac{\Prob(y_i\mid x_i^j, \beta_t, \phi_t) \Prob(x_{gi}^j \mid \gamma_t)}{
\sum_{k}\Prob (y_i\mid x_i^k, \beta_t, \phi_t) \Prob(x_{gi}^k \mid \gamma_t)}
\equiv w_{ij}(\beta_t,\phi_t,\gamma_{gt})  
$$
so that only the distribution $\Prob(x_{gi}^{j} |\ \gamma_t)$
has to be specified.  Such a simplification
is convenient because specifying 
$\Prob(x_{ei}^{j}|\ \gamma_t)$ can be difficult for non-genetic 
covariates that are continuous, for example.

In general, the frequencies of haplotype configurations are not identifiable 
from the genotypes at individual SNPs. For example, haplotype configurations 
resulting in two or more heterozygous SNP genotypes are never observed 
unambiguously. However, under Hardy-Weinberg proportions (HWP), the probability 
of a haplotype configuration can be written in terms of the haplotype frequencies,
allowing both the frequencies of haplotypes and of haplotype configurations to
be estimated from genotypes at individual SNPs. The assumption of HWP holds
if the haplotype inherited from the mother is independent of the haplotype 
inherited from the father. Assuming HWP also leads to a
considerable reduction in the number of parameters $\gamma_g$ 
describing the haplotype covariate distribution $\Prob(x_{gi} \mid \gamma_g)$. 
For example, when considering haplotypes of 3 SNPs, there are $2^{3}=8$ 
possible haplotypes compared to $8(9)/2=36$ possible phase configurations. 
The substantive reduction in the number of genetic parameters leads to 
faster convergence of the EM algorithm and greater stability of estimation.


\item M-step: Maximize the conditional expected log-likelihood $Q(\theta
\mid \theta_t)$ to determine the \mbox{$(t+1)$th} parameter estimate.

Under the assumption of independence of the genetic and non-genetic
covariates, $Q(\theta \mid \theta_t)$ may be re-expressed as a sum of
components involving one of either ($\beta$, $\phi$), $\gamma_g$ or $\gamma_e$:
\begin{eqnarray*}
Q(\theta \mid \theta_t) & = & \sum_i Q_i(\theta \mid \theta_t) 
  \;\;  = \;\; \sum_i \sum_j w_{ij}(\beta_t,\phi_t,\gamma_{gt})
 l_{ci}^j(\theta) \\
  & = & \sum_i \sum_j 
    w_{ij}(\beta_t,\phi_t,\gamma_{gt})
    \log\left[ \Prob(y_i \mid x^j_i, \beta, \phi)
      \Prob(x^j_{gi} \mid \gamma_g) \Prob(x_{ei} \mid \gamma_e) \right] \\
 & = & \sum_i \sum_j 
    w_{ij}(\beta_t,\phi_t,\gamma_{gt})
         \log \Prob(y_i \mid x^j_i, \beta, \phi)\\
 & &+\qquad \sum_i \sum_j 
    w_{ij}(\beta_t,\phi_t,\gamma_{gt})
\log \Prob(x^j_{gi} \mid\gamma_g) \\
& &+\qquad \sum_i \sum_j 
    w_{ij}(\beta_t,\phi_t,\gamma_{gt})
       \log \Prob(x_{ei} \mid \gamma_e) \\
 & \equiv & Q(\beta , \phi \mid \beta_t, \phi_t, \gamma_{gt})+
      Q(\gamma_g \mid \beta_t, \phi_t, \gamma_{gt})+
     Q(\gamma_e \mid \beta_t, \phi_t, \gamma_{gt})
\end{eqnarray*}
Therefore, each component is maximized separately to update the estimates
of $\beta$, $\phi$ and $\gamma_{g}$.

The summand $Q(\beta , \phi\mid \beta_t, \phi_t, \gamma_{gt})$ is a
weighted log-likelihood for regression and dispersion parameters in a
GLM. The \proglang{R} \code{glm} function is used to estimate 
 $\beta_{t+1}$ with the $w_{ij}(\beta_t,\phi_t,\gamma_{gt})$
input to the \code{glm} function through the ``weights'' option. 
The summand $Q(\gamma_g \mid \beta_t, \phi_t, \gamma_{gt})$ is a weighted 
multinomial log-likelihood and so its maximization is also straightforward.
%% as the weighted frequencies of each haplotype. 
Since estimates of $\gamma_e$ are not
required for the weights or for updating estimates of the other
parameters, $\gamma_e$ can be ignored when finding maximum-likelihood
estimates of the regression parameters $\beta$.

\end{enumerate}

\item Repeat E- and M-steps until convergence. 

\item Calculate the standard errors using values from the final GLM.

The variance-covariance matrix of $\widehat{\theta}$ can be approximated
by the inverse of the observed information matrix $I(\widehat{\theta})$.
\citet{Louis82} gives an expression for the observed information in
missing data problems. With ambiguous haplotype configurations, the expression 
consists of a term for the expected complete-data information given the observed
genotypes minus a penalty for unknown phase \citep{Schaid02}.
Factorization of the likelihood implies that only the observed information
$I(\widehat{\beta},\widehat{\phi},\widehat{\gamma_g})$ is required to
obtain standard errors for $\widehat{\beta}$. The variance of
$\widehat{\beta}$ is approximated by the appropriate submatrix of
$I^{-1}(\widehat{\beta},\widehat{\phi},\widehat{\gamma_g})$. Appendix A
provides details of the derivation of
$I^{-1}(\widehat{\beta},\widehat{\phi},\widehat{\gamma_g})$.

\end{enumerate}

\section[Using hapassoc and its features]{Using \pkg{hapassoc} and its features}

Our contributed \proglang{R} package \pkg{hapassoc} 
can be downloaded from the CRAN webpage using the install command at the
\proglang{R} command line (\texttt{>}):

\begin{Schunk}
\begin{Sinput}
> install.packages("hapassoc")
\end{Sinput}
\end{Schunk}

\pkg{hapassoc} does not depend on any optional R packages. After it has 
been installed, the package may be loaded with:

\begin{Schunk}
\begin{Sinput}
> library(hapassoc)
\end{Sinput}
\end{Schunk}

There are four main functions for the user of \pkg{hapassoc}:
\code{pre.hapassoc, hapassoc, summary.hapassoc} 
and \code{anova.hapassoc}. The function
\code{pre.hapassoc} takes the input dataset, with columns of genotype
data, and outputs an augmented dataset consisting of all
pseudo-individuals. The function \code{hapassoc} fits the 
user-specified generalized linear model, with haplotypes as covariates. 
The \code{summary} function provides a summary of the results
while \code{anova} returns a likelihood ratio test of two
nested models fit with \code{hapassoc}.
Further details on each of these functions are now given.

The function \code{pre.hapassoc} pre-processes the input dataset for 
\code{hapassoc}. This pre-processing involves converting the genotype 
data into haplotype data and adding rows to the input dataset 
representing haplotype configurations compatible with the observed 
genotype data for a subject. If the 
haplotype configuration of a subject is known,
no extra rows are added for this subject. 
The pre-processing also involves obtaining initial estimates
for haplotype frequencies based on the genotype data
from all subjects combined.  Haplotypes with non-negligible 
frequency below a user-defined pooling
tolerance (\code{pooling.tol}) are then grouped together.
The function requires the input dataset to have the following form:

\begin{enumerate}

\item Each row represents an individual. Columns represent the trait 
(response), non-genetic covariate information and genotype information.

\item The columns of genotype information may have one of two formats. 
For the ``allelic'' format, each genotype is denoted by two columns, 
one column for each allele, with alleles coded as either 0 or 1. For the
``genotypic'' format, the genotype at each locus is given in one column by
a character string or factor. 
Only two possible variants or alleles are permitted,
forming three possible genotypes. Genotypes with 
more than two alleles will cause \code{pre.hapassoc} to issue an 
error and stop execution. Examples of both allelic and genotypic
formats are given in section \ref{section.examples}.

\item If there are $n$ SNPs, the final $n$ columns of the ``genotypic'' 
dataset or $2n$ columns of the ``allelic'' dataset contain the genotype 
data. By default, \code{pre.hapassoc} prints a list
of genotype variables used to form haplotypes and a list of 
other ``non-genetic'' variables passed to the function.

\item There are no restrictions on the non-genetic covariate columns,
except that missing data should be coded as {\tt NA}. 
Note that only missing
genotype data is handled by \pkg{hapassoc}; individuals with missing
non-genetic data will be removed from the dataset and a warning will be
given to the user.

\item Missing allelic data should be coded as {\tt NA} 
or ` '. 
Missing genotypic data should be coded as, e.g., 
`A', if one allele is missing and the other allele is the 
nucleotide adenine (say), and ` ' or {\tt NA} 
if both alleles are missing.

\end{enumerate}

This function has a number of optional parameters that may be set by the
user. The option \code{maxMissingGenos} specifies the maximum number of 
SNPs which may be missing genotypes for a subject in order
for that subject to be included 
in the analysis, with a default of 1. Each SNP with a missing genotype 
leads to more haplotype configurations being added for that subject in 
the augmented dataset. 
To avoid convergence problems due to haplotypes of low frequency, 
\code{pooling.tol} can be used to specify a threshold frequency below 
which haplotypes will be pooled into a single category. 
The threshold is applied to the initial haplotype
frequencies provided by \code{pre.hapassoc}. The ``pooled'' category
returned by \code{pre.hapassoc} is not updated 
with the haplotype frequencies 
from subsequent iterations of the EM algorithm
in \code{hapassoc}.
The default pooling threshold is set to 0.05. 
Similarly, the parameter \code{zero.tol} 
gives the frequency below which it is assumed that a haplotype does not 
exist. The default threshold is the inverse of ten times the 
total number of haplotypes in all subjects. Pseudo-individuals carrying 
haplotypes of estimated frequency below this zero threshold will be removed 
from the augmented dataset.
The \code{verbose} flag, if \code{TRUE} (the default),
causes \code{pre.hapassoc} to print a list of the genotype
variables used to form haplotypes and a list of other non-genetic
variables.

The function \code{pre.hapassoc} will return a list with components 
required for \code{hapassoc}. The components include two data frames,
\code{haploMat} and \code{haploDM}, with the haplotype information. 
The data frame \code{haploMat} consists of two columns, one for each of the 
inferred haplotypes, and a row for each pseudo-individual in the dataset. 
The data frame \code{haploDM} consists of a column for each of the haplotypes 
inferred to be present in the dataset and a row for each pseudo-individual. 
Each pseudo-individual is given a code of 0, 1 or 2 in each column for the 
number of copies they have of that haplotype. Therefore, all  
rows of \code{haploDM} should sum to 2, as each individual can have 
only two haplotypes. The non-haplotype data is returned in the data frame 
\code{nonHaploDM}. Some rows of \code{nonHaploDM} may be identical, as they 
represent the trait value and non-genetic covariate information for the same 
individual who has multiple haplotype configurations compatible with
their observed genotype data. Other components of the list returned by 
\code{pre.hapassoc} include a vector \code{initFreq} of initial estimates of 
all haplotype frequencies (including pooled haplotypes), vectors \code{zeroFreqHaplos} 
and \code{pooledHaplos} of the zero-frequency and pooled haplotypes, 
respectively, and a vector \code{wt} of the initial estimates for the 
weights. For more information on this function, type

\begin{Schunk}
\begin{Sinput}
> help(pre.hapassoc)
\end{Sinput}
\end{Schunk}

The function \code{hapassoc} uses the list returned from 
\code{pre.hapassoc} as input, and fits the linear model provided in the
model-formula argument. The model formula is specified the same way as it
would be for the \code{glm} or \code{lm} functions in \proglang{R}. 
The elements of the model formula must have names from among the names of 
columns in the data frames \code{nonHaploDM} and \code{haploDM}.
The availability of \code{haploDM}
provides the user with the flexibility to specify recessive, dominant
or other types of genetic risk models.
See the examples in section \ref{section.examples} below for more details 
on specifying the model equation. As with \code{glm}, the family of the 
GLM is specified with the family parameter. Currently 
the \code{binomial}, \code{gaussian}, \code{poisson},
and \code{Gamma} families are supported by \code{hapassoc}. 
Unlike \code{glm}, we have opted for a \code{binomial}
family as the default,
since in most of the biomedical applications 
we have encountered 
binary traits (e.g. diseased/not diseased) have been investigated.
Other optional parameters 
can be set including the initial estimates of haplotype frequencies and of 
regression coefficients used to start the EM algorithm, the maximum number 
of iterations for the algorithm,
the tolerance for convergence,
and a \code{verbose} flag that controls 
printing of the iteration number and the value of 
the covergence criterion at each iteration.

The function returns a list of class ``hapassoc" with components such as the 
estimated regression coefficients, the estimated haplotype frequencies and an 
estimate of their joint variance-covariance matrix. The associated
help file, found by typing \texttt{help(hapassoc)},
provides a detailed description of the components
of the returned list.
The function \code{summary.hapassoc} is the 
summary method for the class and provides a nicer way of viewing the
results. The summary includes a table of the estimated regression 
coefficients and the associated standard errors, Z scores and p-values, 
a table of the estimated haplotype frequencies and their standard errors, 
and the estimated dispersion parameter (when appropriate).
Likelihood ratio tests to compare
two nested models fit with \code{hapassoc} may be performed with
\code{anova.hapassoc}. 

\section{Examples} \label{section.examples}

This section illustrates the use of \code{hapassoc} with examples of logistic 
and linear regression, when input genotypes are in ``allelic" and ``genotypic" 
format, respectively.

\subsection{Logistic regression with input genotypes in ``allelic'' format}

The ``hypoDat'' dataset from the \code{hapassoc} package will be used 
for this example. The dataset can be loaded into \proglang{R} with the command

\begin{Schunk}
\begin{Sinput}
> data(hypoDat)
\end{Sinput}
\end{Schunk}

The first five rows of the dataset are shown
\begin{Schunk}
\begin{Sinput}
> hypoDat[1:5, ]
\end{Sinput}
\begin{Soutput}
  affected       attr M1.1 M1.2 M2.1 M2.2 M3.1 M3.2
1        0 -1.0457877    0    0    0    1    1    1
2        0 -1.3604013    0    0    0    0    0    1
3        0  0.5672941    0    0    0    1    0    1
4        0 -0.2220068    1    1    0    0    0    0
5        0  1.8051683    0    0    0    1    0    1
\end{Soutput}
\end{Schunk}
The data consists of eight columns, the affection status (0/1), a continuous 
non-genetic covariate, and information on three diallelic genetic markers. 
The genotype data is in ``allelic'' format, so there are six genotype columns 
for the three SNPs. The genotype of the first SNP is represented by the two 
columns M1.1 and M1.2. For example, the fourth individual has genotype 
1/1 (two copies of allele 1) at the first SNP.

The genotype data is converted to haplotype data by the \code{pre.hapassoc} 
function. Data frames passed to \code{pre.hapassoc}
are assumed to contain the 
response, an arbitrary number of non-genetic covariates and the 
genotype data. The user must tell the function explicitly how to 
separate the genetic and 
non-genetic data by specifying \code{numSNPs} and by passing a data frame 
whose last 2$\times$\code{numSNPs} (allelic format) or last \code{numSNPs}
(genotypic format) columns comprise the data on genotypes. The data frame
\code{hypoDat} is in allelic format and so the appropriate call is
\begin{Schunk}
\begin{Sinput}
> example1.haplos <- pre.hapassoc(hypoDat, numSNPs = 3)
\end{Sinput}
\begin{Soutput}
Haplotypes will be based on the following SNPs (allelic format): 
 SNP 1: M1.1/M1.2
 SNP 2: M2.1/M2.2
 SNP 3: M3.1/M3.2
Remaining variables are: 
 affected, attr 
\end{Soutput}
\end{Schunk}
As an aside, if only the first and third
SNPs were of interest for haplotype effects, the call to  
\code{pre.hapassoc} could be modified to
\begin{Schunk}
\begin{Sinput}
> example2.haplos <- pre.hapassoc(hypoDat[, c(1:4, 7:8)], numSNPs = 2)
\end{Sinput}
\begin{Soutput}
Haplotypes will be based on the following SNPs (allelic format): 
 SNP 1: M1.1/M1.2
 SNP 2: M3.1/M3.2
Remaining variables are: 
 affected, attr 
\end{Soutput}
\end{Schunk}

Similarly, if only the first two SNPs were of interest the call would be
\begin{Schunk}
\begin{Sinput}
> example2.haplos <- pre.hapassoc(hypoDat[, 1:6], numSNPs = 2)
\end{Sinput}
\begin{Soutput}
Haplotypes will be based on the following SNPs (allelic format): 
 SNP 1: M1.1/M1.2
 SNP 2: M2.1/M2.2
Remaining variables are: 
 affected, attr 
\end{Soutput}
\end{Schunk}

Finally, if the last two SNPs were of interest the possible calls are
\begin{Schunk}
\begin{Sinput}
> example2.haplos <- pre.hapassoc(hypoDat[, c(1:2, 5:8)], numSNPs = 2)
\end{Sinput}
\begin{Soutput}
Haplotypes will be based on the following SNPs (allelic format): 
 SNP 1: M2.1/M2.2
 SNP 2: M3.1/M3.2
Remaining variables are: 
 affected, attr 
\end{Soutput}
\begin{Sinput}
> example2.haplos <- pre.hapassoc(hypoDat, numSNPs = 2)
\end{Sinput}
\begin{Soutput}
Haplotypes will be based on the following SNPs (allelic format): 
 SNP 1: M2.1/M2.2
 SNP 2: M3.1/M3.2
Remaining variables are: 
 affected, attr, M1.1, M1.2 
\end{Soutput}
\end{Schunk}

We prefer the first of these calls since it 
will exclude the data on
the first SNP, \code{M1.1} and \code{M1.2}, from the non-haplotype 
data frame \code{nonHaploDM}
output by \code{pre.hapassoc}.  The second call includes the 
columns \code{M1.1} and \code{M1.2} describing the first SNP
in \code{nonHaploDM}.

The first two individuals in the \code{hypoDat} dataset 
have haplotype configurations that can be inferred with certainty. 
However, the haplotype configuration of the third individual 
is uncertain. This subject either has haplotypes 000/011 or
001/010. The uncertain haplotype phasing for this subject is reflected 
in the data frames \code{haploDM} and \code{nonHaploDM} 
returned by \code{pre.hapassoc}. The rows of \code{haploDM} corresponding 
to the first five rows of \code{hypoDat} are
\begin{Schunk}
\begin{Sinput}
> example1.haplos$haploDM[1:7, ]
\end{Sinput}
\begin{Soutput}
  h000 h001 h010 h011 h100 pooled
1    0    1    0    1    0      0
2    1    1    0    0    0      0
3    1    0    0    1    0      0
4    0    1    1    0    0      0
5    0    0    0    0    2      0
6    1    0    0    1    0      0
7    0    1    1    0    0      0
\end{Soutput}
\end{Schunk}
Rows three and four of \code{haploDM} correspond to row three of \code{hypoDat}
for the third subject with the uncertain haplotype configuration. 
The column for the ``pooled'' category of \code{haploDM} is comprised 
of haplotypes 101, 110 and 111 (see the estimated haplotype frequencies 
below). The rows of \code{nonHaploDM} corresponding to the first five 
subjects in \code{hypoDat} are:

\begin{Schunk}
\begin{Sinput}
> example1.haplos$nonHaploDM[1:7, ]
\end{Sinput}
\begin{Soutput}
  affected       attr
1        0 -1.0457877
2        0 -1.3604013
3        0  0.5672941
4        0  0.5672941
5        0 -0.2220068
6        0  1.8051683
7        0  1.8051683
\end{Soutput}
\end{Schunk}
As with \code{HaploDM}, rows three and four of \code{nonHaploDM} 
are pseudo-individuals 
that correspond to the subject in row three of \code{hypoDat}, and therefore 
have the same trait and non-genetic covariate data.

The initial estimates of haplotype frequencies are
\begin{Schunk}
\begin{Sinput}
> example1.haplos$initFreq
\end{Sinput}
\begin{Soutput}
      h000       h001       h010       h011       h100       h101       h110 
0.25179111 0.26050418 0.23606001 0.09164470 0.10133627 0.02636844 0.01081260 
      h111 
0.02148268 
\end{Soutput}
\end{Schunk}
Haplotypes h101, h110, and h111 all have frequencies 
below the default pooling threshold of 0.05 and so they 
comprise the pooled category in
\code{example1.haplos\$haploDM}. 

Logistic regression is now performed to determine the effects of haplotypes 
on affection status, after adjusting for the non-genetic covariate ``attr''.

\begin{Schunk}
\begin{Sinput}
> example1.regr1 <- hapassoc(affected ~ attr + h000 + h010 + h011 + 
+     h100 + pooled, example1.haplos, family = binomial())
\end{Sinput}
\end{Schunk}

The same regression can be fit with 
\begin{Schunk}
\begin{Sinput}
> example1.regr <- hapassoc(affected ~ ., baseline = "h001", example1.haplos, 
+     family = binomial())
\end{Sinput}
\end{Schunk}
where the ``.''
in the model formula
is an \proglang{R} short form for the other columns in the
data matrix. Haplotype h001 was chosen as the baseline haplotype
in this example because it is the most frequent (the default
in \code{hapassoc}). Users are free to choose their own 
baseline haplotype, but are cautioned that using a rare haplotype (or the
pooled category, if rare) as the baseline can lead to unstable
parameter estimates.

Other formula syntax is supported in the hapassoc function.
For example,
\begin{Schunk}
\begin{Sinput}
> example1.regr <- hapassoc(affected == 0 ~ attr + h000 + h010 + 
+     h011 + h100 + pooled, example1.haplos, family = binomial())
\end{Sinput}
\end{Schunk}
can be used to specify that the probability that affection status is 0 is
modelled, rather than 1. This feature can be useful if the binary outcome 
is specified by character strings, such as ``yes''/``no'', since the 
formula may then be specified as \code{hapassoc(affected == 'yes' ~ ...)}

Additionally, different models for haplotype effects can be fit using
\proglang{R} formula syntax. The default is an additive effect for each 
copy of the haplotype on the scale of the linear predictor. The following 
command fits recessive effects for haplotypes h000 and h001 
(i.e., the haplotype has an effect only if the individual has two copies).
\begin{Schunk}
\begin{Sinput}
> example1.regr2 <- hapassoc(affected ~ attr + I(h000 == 2) + I(h001 == 
+     2), example1.haplos, family = binomial())
\end{Sinput}
\end{Schunk}
The output of \code{hapassoc} may be summarized with:
\begin{Schunk}
\begin{Sinput}
> summary(example1.regr)
\end{Sinput}
\begin{Soutput}
Call:
hapassoc(form = affected == 0 ~ attr + h000 + h010 + h011 + h100 + 
    pooled, haplos.list = example1.haplos, family = binomial())

Number of subjects used in analysis: 100 

Coefficients:
               Estimate Std. Error      zscore   Pr(>|z|)
(Intercept)  1.24304712  0.7826121  1.58833114 0.11221148
attr        -0.74117082  0.2919206 -2.53894658 0.01111868
h000        -1.15178187  0.5942128 -1.93833229 0.05258270
h010         0.59528765  0.6570407  0.90601329 0.36492882
h011         0.03426242  0.9155250  0.03742379 0.97014710
h100         0.85628090  1.0199383  0.83954187 0.40116530
pooled      -0.38745841  0.8775181 -0.44153893 0.65882289

Estimated haplotype frequencies:
         Estimate Std. Error
f.h000 0.26732097 0.03930403
f.h001 0.25180046 0.03865720
f.h010 0.21981267 0.03877367
f.h011 0.10106590 0.02949351
f.h100 0.09501225 0.02369630
f.h101 0.02586632 0.01412337
f.h110 0.01785410 0.01384235
f.h111 0.02126733 0.01247509

(Dispersion parameter for binomial family taken to be 1)

Log-likelihood: -322.155788349086 
\end{Soutput}
\end{Schunk}
A global hypothesis test for haplotype effects may be performed with:
\begin{Schunk}
\begin{Sinput}
> example1.regr2 <- hapassoc(affected ~ attr, example1.haplos, 
+     family = binomial())
> anova(example1.regr1, example1.regr2)
\end{Sinput}
\begin{Soutput}
	hapassoc: likelihood ratio test

Full model: affected ~ attr + h000 + h010 + h011 + h100 + pooled 
Reduced model: affected ~ attr 

LR statistic = 11.2107 , df = 5 , p-value =  0.0474 
\end{Soutput}
\end{Schunk}

\subsection{Linear regression with input genotypes in ``genotypic'' format}

An example of the ``genotypic" input format is found in the dataset 
\code{hypoDatGeno}. The first five rows are shown below.

\begin{Schunk}
\begin{Sinput}
> data(hypoDatGeno)
> hypoDatGeno[1:5, ]
\end{Sinput}
\begin{Soutput}
  affected       attr M1 M2 M3
1        0 -1.0457877 AA AC CC
2        0 -1.3604013 AA AA AC
3        0  0.5672941 AA AC AC
4        0 -0.2220068 CC AA AA
5        0  1.8051683 AA AC AC
\end{Soutput}
\end{Schunk}
In this case, the alleles at all three SNPs are now represented with A's
and C's rather than 0's and 1's. Additionally, the genotype at each 
SNP is given in one column.

To illustrate how \code{hapassoc} handles missing genotype data,
we introduce missing genotypic information for SNPs M1 and M2,
as shown below. We modify the first subject's genotype at M1
so that it is completely missing. We modify the second subject's
genotype at M2 so that only one allele, an `A', is known.
Since the data frame \code{hypoDataGeno} is in genotypic format,
M2 is a factor.  Therefore, before modifying the second subject's M2
genotype, we must add a level `A' to M2:
\begin{Schunk}
\begin{Sinput}
> hypoDatGeno[1, "M1"] <- NA
> levels(hypoDatGeno[, "M2"]) <- c(levels(hypoDatGeno[, "M2"]), 
+     "A")
> hypoDatGeno[2, "M2"] <- "A"
\end{Sinput}
\end{Schunk}

The data in \code{hypoDatGeno} is now pre-processed by \code{pre.hapassoc}. 
For illustration, the optional parameter \code{maxMissingGenos} 
has been changed from its default value of one to two.

\begin{Schunk}
\begin{Sinput}
> example2.haplos <- pre.hapassoc(hypoDatGeno, 3, maxMissingGenos = 2, 
+     allelic = F)
\end{Sinput}
\begin{Soutput}
Haplotypes will be based on the following SNPs (genotypic format): 
 M1, M2, M3 
Remaining variables are: 
 affected, attr 
\end{Soutput}
\begin{Sinput}
> example2.haplos$nonHaploDM[142:148, ]
\end{Sinput}
\begin{Soutput}
    affected      attr
142        0 -1.045788
143        0 -1.045788
144        0 -1.045788
145        0 -1.045788
146        0 -1.360401
147        0 -1.360401
148        0 -1.360401
\end{Soutput}
\end{Schunk}

When there are no more than \code{maxMissingGenos} SNPs with missing 
genotype data, rows corresponding to haplotype configurations compatible
with the missing genotype data for a subject are added to the end of the 
augmented data matrices \code{haploDM} and \code{nonhaploDM} returned by 
\code{pre.hapassoc}. For example, rows 142 to 145 of \code{nonHaploDM} 
correspond to the first subject in \code{hypoDatGeno} and rows 146 to 148
of \code{nonHaploDM} correspond to the second subject in \code{hypoDatGeno}.
When there are more missing genotypes than 
are allowed by the user, as in the following command, a warning 
message is given to indicate that some individuals have been removed from 
the dataset. 

\begin{Schunk}
\begin{Sinput}
> pre.hapassoc(hypoDatGeno, 3, maxMissingGenos = 0, allelic = F, 
+     verbose = FALSE)
\end{Sinput}
\end{Schunk}
%% Have to manually insert the Warning message
\begin{Code}
Warning message:
2 subjects with missing data in more than 0 genotype(s) removed
 in: handleMissings(SNPdat, nonSNPdat, numSNPs, maxMissingGenos)
\end{Code}


With SNP data in genotypic format, the haplotypes are now denoted by 
combinations of the A and C alleles at each SNP, as can be seen 
in the display of the initial haplotype frequency estimates.
\begin{Schunk}
\begin{Sinput}
> example2.haplos$init
\end{Sinput}
\begin{Soutput}
      hAAA       hAAC       hACA       hACC       hCAA       hCAC       hCCA 
0.24859595 0.26034040 0.23949970 0.09008737 0.10150449 0.02672430 0.01039986 
      hCCC 
0.02284793 
\end{Soutput}
\end{Schunk}

To illustrate linear regression, the continuous non-genetic covariate 
{\tt attr} will be used as a continuous trait in the call to \code{hapassoc}:

\begin{Schunk}
\begin{Sinput}
> example2.regr <- hapassoc(attr ~ hAAA + hACA + hACC + hCAA + 
+     pooled, example2.haplos, family = gaussian())
> summary(example2.regr)
\end{Sinput}
\begin{Soutput}
Call:
hapassoc(form = attr ~ hAAA + hACA + hACC + hCAA + pooled, haplos.list = example2.haplos, 
    family = gaussian())

Number of subjects used in analysis: 100 

Coefficients:
                Estimate Std. Error      zscore  Pr(>|z|)
(Intercept) -0.198910820  0.3145337 -0.63239909 0.5271261
hAAA        -0.003337329  0.2422522 -0.01377626 0.9890085
hACA        -0.005089539  0.2213682 -0.02299129 0.9816572
hACC         0.423914584  0.4979658  0.85129263 0.3946068
hCAA         0.300230129  0.2938526  1.02170329 0.3069214
pooled       0.019130817  0.3311376  0.05777302 0.9539294

Estimated haplotype frequencies:
          Estimate Std. Error
f.hAAA 0.248400660 0.03590094
f.hAAC 0.257565353 0.03806208
f.hACA 0.241325932 0.03550967
f.hACC 0.090867626 0.02673329
f.hCAA 0.101119710 0.02355280
f.hCAC 0.030216413 0.01611531
f.hCCA 0.009153699 0.00911681
f.hCCC 0.021350608 0.01213278

(Dispersion parameter for gaussian family taken to be 1.00958412775600)

Log-likelihood: -410.627503879034 
\end{Soutput}
\end{Schunk}
The \code{dispersion} reported by the summary function
is the maximum likelihood estimate of the error variance in the linear
model.

\section{Summary}

\pkg{hapassoc} is an \proglang{R} package for haplotype-trait associations
in the presence of uncertain haplotype phase, for dichotomous or
continuous traits. It uses an EM algorithm called the method of weights to
handle the missing haplotype-phase information. Non-genetic attributes, 
environmental covariates and haplotype-environment interactions can be 
included in generalized linear models of trait associations. Assumptions 
of Hardy-Weinberg proportions and independence of genetic and non-genetic 
covariates are made.  In addition to additive effects, dominant and recessive 
effects of the haplotype risk factors can also be specified in \pkg{hapassoc}, 
in contrast to similar haplotype-trait association software. 
Missing genotype data
is also handled by \pkg{hapassoc} so that those 
individuals with a limited amount 
of missing genotype data are not removed during the analysis. 

\pkg{hapassoc} relies on many of the \proglang{R} base functions, and for that
reason it uses much of the same syntax. For someone familiar with the
\proglang{R} \code{glm} and \code{lm} functions, using \pkg{hapassoc}
should be relatively straightforward, as the specification of 
model formulae is 
identical, the output of the \code{summary} method is 
similar and the \code{anova} method is similar
when used to compare two nested models. 
Currently \code{anova.hapassoc} differs from the \code{anova} methods
for \code{lm} and \code{glm} in that \code{anova.hapassoc} requires
exactly two models as input. By contrast, \code{anova.lm} and 
\code{anova.glm} take either one fitted model, in which case a 
sequence of possible sub-models are compared, or 
an arbitrary number of nested models.
Work to make \code{anova.hapassoc} behave more like \code{anova.lm}
and \code{anova.glm} is underway and will
be included in a future version of \pkg{hapassoc}. 
As an alternative to likelihood ratio tests, users may construct the
appropriate Wald statistics in the usual way, with the 
estimated coefficients and variance-covariance matrix returned by the 
\code{hapassoc} function.


\section*{Acknowledgements}

We would like to thank Matthew Pratola for optimizing the R code and 
initiating its conversion to an R package, and Sigal Blay for 
code optimization and package maintenence. We would also like to
thank the \pkg{hapassoc} users for their valuable feedback 
in bug reports. This research was supported
by Natural Sciences and Engineering Research Council of Canada grants
227972-00 and 213124-03,
by Juvenile Diabetes Research Foundation International grant 1-2001-873,
by Canadian Institutes of Health Research grants 
NPG-64869, ATF-66667 and GEI-53960, and in part by
the Mathematics of Information Technology and Complex Systems,
Canadian National Centres of Excellence.
JG is a Scholar of the BC Michael Smith Foundation for Health Research.

\bibliography{burkett_biblio} 

\section*{Appendix: Details on the computation of the standard errors}

>From standard likelihood theory
\citep[e.g.][]{McCullaghNelder83}, 
the variance-covariance
matrix of the maximum likelihood estimators $\widehat{\theta}$
can be approximated by
the inverse of the Fisher information matrix
$$ {\cal I}(\theta_0) = \E\left[ I(\theta_0) \right]
\equiv \E\left[ - \frac{\partial^2}{\partial \theta \partial \theta^T}
l(\theta_0) \right] $$
where $\theta_0$ is the ``true'' value of $\theta$ and $l(\theta)$ is
the observed-data log-likelihood. The usual estimate of
$ {\cal I}(\theta_0)$ is the observed information $I(\widehat{\theta})$.

Under independence of genetic and non-genetic covariates,
the observed-data likelihood
$L(\theta)$ factors into a term
involving $\beta$, $\phi$ and $\gamma_g$ and another term involving
$\gamma_e$:
\begin{eqnarray*}
L(\theta)  =  \prod_i \Prob(y_i , x_{obs,i} \mid \theta)
& = & \prod_i \sum_j  \Prob(y_i , x^j_i \mid \theta) \\
& = & \prod_i \sum_j  \Prob(y_i \mid x^j_i, \beta, \phi)
      \Prob(x^j_{gi} \mid \gamma_g) \Prob(x_{ei} \mid \gamma_e)  \\
& = & \prod_i \Prob(x_{ei} \mid \gamma_e) \sum_j  
      \Prob(y_i \mid x^j_i, \beta,\phi)
      \Prob(x^j_{gi} \mid \gamma_g)  \\
& = & \left[ \prod_i \sum_j  \Prob(y_i \mid x^j_i, \beta,\phi)
      \Prob(x^j_{gi} \mid \gamma_g) \right] \times
      \prod_i \Prob(x_{ei} \mid \gamma_e)
\end{eqnarray*}
The observed information $I(\theta)$
is therefore block-diagonal:
$$
I(\theta) =
\left[ \begin{array}{ll} I(\beta,\phi,\gamma_g) & 0 \\
                                    0 & I(\gamma_e) \\
       \end{array} \right],
\qquad \mbox{and so}
\qquad
I^{-1}(\theta) =
\left[ \begin{array}{ll} I^{-1}(\beta,\phi,\gamma_g) & 0 \\
                                    0 & I^{-1}(\gamma_e) \\
       \end{array} \right].
$$
Thus, only $I(\beta,\phi,\gamma_g)$
is required for inference of $\beta$.


\subsection*{Information matrix for  $\beta$, $\phi$ and $\gamma_g$}

\citet{Louis82} derived an expression for the observed information
in missing data problems in terms of complete-data score vectors and 
information matrices.
Let $S_c(\beta,\phi,\gamma_g)$ and $S_{ci}(\beta,\phi, \gamma_g)$
be the complete-data score functions for all subjects
and for the $i$th subject, respectively.
Define $I_c(\beta, \phi, \gamma_g)$ and $I_{ci}(\beta, \phi, \gamma_g)$
to be the corresponding observed information matrices.
Further, let $S^j_{ci}(\beta,\phi,\gamma_g)$
and $I^j_{ci}(\beta, \phi, \gamma_g)$
be, respectively, the complete-data
score function and observed information
for the $i$th subject with covariate vector $x^j_i$.
Under the linear model,
recall $\eta^j_i = x^j_i \, ^T \beta$ and $\mu^j_i=g^{-1}(\eta^j_i)=
b'(\nu^j_i)$. Define $d_{ij}= y_i \nu_i^j - b(\nu_i^j)$ and
let $c'(y,\phi)$ and $c''(y,\phi)$ denote
the first and second derivatives of $c(y, \phi)$ with respect to $\phi$.
Then $S^j_{ci}(\beta,\phi,\gamma_g)^T =
   \left[ S^j_{ci}(\beta)^T,S^j_{ci}(\phi),S^j_{ci}(\gamma_g)^T\right]$
where
\begin{eqnarray*}
S^j_{ci}(\beta) & = &
      \frac{\partial}{\partial \beta} \log P(y_i \mid x^j_i, \beta,\phi)
\; = \; \frac{m_i(y_i - \mu^j_i)}{\phi} \; x^j_i, \\
S^j_{ci}(\phi) & = &
      \frac{\partial}{\partial \phi} \log P(y_i \mid x^j_i, \beta,\phi)
\; = \; -\frac{m_i d_{ij}}{\phi^2} + c'(y_i,\phi) \\
\mbox{and} \quad S^j_{ci}(\gamma_g) &= &
      \frac{\partial}{\partial \gamma_g} \log P(x^j_i \mid \gamma_g).
\end{eqnarray*}

The observed information
$I^j_{ci}(\beta, \phi, \gamma_g)$
has three submatrices along its diagonal.
In the top left-hand corner,
$$
     I^j_{ci}(\beta,\beta)  =
     - \frac{\partial^2}{\partial \beta \partial \beta^T}
               \log P(y_i \mid x^j_i, \beta,\phi) =
\frac {m_i \, x^j_i \, {x^j_i}^{\,T}} {\phi \, g'( \mu^j_i)},
$$
where
$g'(\mu^j_i)$ is the derivative of the link function
with respect to $\mu$,
evaluated at $\mu^j_i$.
The next diagonal submatrix is
$$
     I^j_{ci}(\phi,\phi)=
     -\frac{\partial^2}{\partial \phi \partial \phi}
               \log P(y_i \mid x^j_i, \beta,\phi) =
-\frac{2 m_i d_{ij}}{\phi^3} - c''(y_i,\phi).$$
Finally, in the bottom right-hand corner
$$
\quad I^j_{ci}(\gamma_g,\gamma_g)=
     -\frac{\partial^2}{\partial \gamma_g \partial \gamma_g^T}
               \log P(x^j_i \mid \gamma_g).
$$
The three off-diagonal submatrices
are $I^j_{ci}(\beta,\gamma_g) = I^j_{ci}(\phi,\gamma_g) = 0$
and
$$
     I^j_{ci}(\beta,\phi)  =
     -\frac{\partial^2}{\partial \beta \partial \phi}
               \log P(y_i \mid x^j_i, \beta,\phi).
$$

The expressions for
 $S^j_{ci}(\gamma_g)$ and $I^j_{ci}(\gamma_g,\gamma_g)$ depend on the
distribution of genetic covariates.
For example, suppose that the $j$th
possible haplotype configuration consistent with
the observed SNP genotypes of the $i$th subject has
$x_{gi}^j = (n_1, \ldots, n_{H+1}),$ where
$n_h=0,1,2$ codes the number of copies of haplotype $h$
and there are $H+1$ haplotypes in the population.
Then, under HWP,
$$\Prob(x_{gi}^j | \gamma_g) \propto
{\gamma_{g1}}^{n_1} \times
{\gamma_{g2}}^{n_2} \times \cdots \times
{\gamma_{gH}}^{n_H} \times
\left( 1- \sum_{h=1}^{H} \gamma_{gh} \right ) ^{n_{H+1}},$$
where $\gamma_{gh}$ is the population frequency of haplotype $h$.
The score vector is
$$
S^j_{ci}(\gamma_g) = \left ( \frac{n_1}{\gamma_{g1}}, \;\; \cdots \;\;,
\frac{n_H}{\gamma_{gH}} \right )^T
- \frac{n_{H+1}} {1- \sum_{h=1}^{H} \gamma_{gh}}
\;\left ( 1, \cdots, 1 \right )^T
$$ 
$$
\mbox{and} \quad I^j_{ci}(\gamma_g,\gamma_g) = -{\rm diag} \left (
\frac{n_1}{\gamma_{g1}^2},
\cdots,
\frac{n_{H}}{\gamma_{gH}^2} \right ) -
\frac{n_{H+1}}{ \left(1- \sum_{h=1}^{H} \gamma_{gh}\right )^2} \; J,
$$
where $J$ is an $H \times H$ matrix of ones.

\citet{Louis82} writes the observed information as
$$
       I(\beta, \phi, \gamma_g) =
{\rm E}[I_c(\beta, \phi, \gamma_g ) | x_{obs}, y] -
                       {\rm Var}[S_c(\beta,\phi,\gamma_g) | x_{obs}, y].
$$
Under independence of subjects, it follows that
$$
I(\beta, \phi, \gamma_g) =
 \sum_i \E[I_{ci}(\beta, \phi, \gamma_g ) | x_{obs,i}, y_i] -
         \sum_i \VAR[S_{ci}(\beta,\phi, \gamma_g) | x_{obs,i}, y_i]
$$
where
\begin{equation}
\E[I_{ci}(\beta,\phi, \gamma_g ) | x_{obs,i}, y_i] \
     = \sum_j w_{ij}(\beta,\phi,\gamma_g) I^j_{ci}(\beta,\phi, \gamma_g )
\label{eq_Louis1}
\end{equation}
and
\begin{eqnarray*}
\VAR[S_{ci}(\beta,\phi,\gamma_g) | x_{obs,i}, y_i] &=&
\sum_j w_{ij}(\beta,\phi,\gamma_g) S^j_{ci}(\beta,\phi,\gamma_g)
                              S^j_{ci}(\beta,\phi,\gamma_g)^T  - \\
& &  \qquad \left\{ \sum_k w_{ik}(\beta,\phi,\gamma_g)
S^k_{ci}(\beta,\phi,\gamma_g) \right\}
    \left\{ \sum_l w_{il}(\beta,\phi,\gamma_g)
S^l_{ci}(\beta,\phi,\gamma_g)^T \right\}
\end{eqnarray*}
\citep{LipsitzIbrahim96, Burkett02}

It can be shown that the expected value of
$ \sum_j w_{ij}  I^j_{ci}(\beta,\phi)$ is a vector of zeroes.
Thus, to estimate the Fisher information,
we may replace the $I^j_{ci}(\beta,\phi)$ submatrix of
$I^j_{ci}(\beta,\phi,\gamma_g)$
in
equation~(\ref{eq_Louis1}) by a vector of zeros and take
$I^j_{ci}(\beta, \phi, \gamma_g)$ to be a block-diagonal
matrix.
%$$I^j_{ci}(\beta, \phi, \gamma_g)
% = \left[
%      \begin{array}{lll}   I^j_{ci}(\beta,\beta) & 0 & 0 \\
%                           0 & I^j_{ci}(\phi,\phi) & 0  \\
%                           0 &  0 &   I^j_{ci}(\gamma_g,\gamma_g)
%\end{array}
%      \right].
%   $$

\end{document}
